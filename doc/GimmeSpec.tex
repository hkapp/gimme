\documentclass[10pt,a4paper]{article}

\usepackage[utf8]{inputenc}
\usepackage[english]{babel}
\usepackage{amsmath}
\usepackage{amsfonts}
\usepackage{amssymb}
\usepackage[left=1in,right=1in,top=1in,bottom=1in]{geometry}

\usepackage{listings}

\newcommand{\gimme}{\textsf{gimme}}

\title{Gimme:\\ A simple versioning system for shared folders}

\begin{document}
\maketitle

\section{Introduction}

We present here \gimme, a simple versioning system for shared folders. It uses an assymetric model, in which one of the parties acts as the "master" or "real truth", and the other is a dirty local copy.

\subsection{Possible names}

The names to be used to reference the two parties involved are not fixed yet. Possible pairs include :
\begin{itemize}
\item master version vs local copy
\item truth vs dirty copy
\item remote vs local
\item shared vs local
\item accepted version vs pending changes
\end{itemize}

\section{Model}

We describe here the model used by \gimme{}.

\subsection{Master version vs local copy}

\gimme{} uses an assymetric model, where one of the two parties involved acts as a "master", and the other is seen as a local copy.

The master version holds the shared truth for all the systems. It holds all the information, in the latest \textit{accepted} state. Information (conceptually speaking) should never disappear from the master version. its goal is to aggregate all the information present on all the systems.

The local copy is a working directory for a single machine. Files can be modified, copied, added, removed, etc... without affecting the master version. In this case, the copy is called \textit{dirty}.

Changes to the local copy are shared with all systems by calling \gimme. These changes then become part of the shared truth. We say these changes are \textit{accepted}. This is called an \textit{upload}, or \textit{update}.

The other way around, changes that are part of the shared truth but are not present in the local copy are also synchronized upon calling \gimme. This process is called \textit{download} or \textit{upgrade}.

\subsection{Subscription model}

One of the big differences between the master version and local copies is that local copies do not necessarily have all the information, and maybe they do not want it.

We model this using \textit{subscriptions}. Local copies explicitely subscribe to a specific set of folders and files. A local copy is not interested in updates or removal of folders or files it does not subscribe to. When a new folder or file is added in a folder that the local copy is subscribed to, the user it prompted to know whether this copy should subscribe to this new element.

It is still unclear what the exact granularity of subscriptions is. It should be possible to choose which folders to subscribe to, and which not too. There could be a mechanism to tell that a folder is subscribed to all its (future) content. This can be bound to the folder, or only to the current update.

The only reason to have a granularity at the level of single files would be to allow deleting a file locally because it is not necessary, but keeping it in the master version. That use-case seems very obscure. When speaking about music files, a similar result can be achieved by removing the file from the library.

To subscribe, the folder only needs to be present in the local copy. New folders added in the master version can be proposed for copy in the local copy, but this process should not be the default. It is always possible to copy a folder from the master file system to subscribe to it.

To unsubscribe, the folder needs to be deleted on the local copy. Then, when computing the changelog, the user should tell explicitely that this deletion is an unsubscribe, and should not be reflected in the master version. There is no default between the two behaviours, the choice must be made explicit anyway.

\subsection{Data considered}

We consider folders and files.

\subsection{Possible actions}

For a given element (either file or folder), the possible actions can be :
\begin{enumerate}
\item Add
\item Delete
\item Update
\end{enumerate}

Note that folders can only be updated if their content has been modified (any of the changes listed above).

\subsection{State machine}

\newcommand{\logstate}[1]{\textsc{#1}}
\newcommand{\sub}[0]{\logstate{sub}}
\newcommand{\unsub}[0]{\logstate{unsub}}
\newcommand{\subrm}[0]{\logstate{rm}}
\newcommand{\track}[0]{\logstate{add}}
\newcommand{\trackrm}[0]{\logstate{rm}}

\newcommand{\physstate}[1]{\textsc{#1}}
\newcommand{\same}[0]{\physstate{sam}}
\newcommand{\modif}[0]{\physstate{upd}}
\newcommand{\present}[0]{\physstate{add}}
\newcommand{\absent}[0]{\physstate{rm}}

\newcommand{\action}[1]{\textsf{#1}}
\newcommand{\donothing}{\action{nothing}}
\newcommand{\cpfile}{\action{cp}}
\newcommand{\cplatest}{\action{latest}}
\newcommand{\metaupd}{\action{upd-hash}}
\newcommand{\rmfile}{\action{rm}}
\newcommand{\mergefolders}{\action{merge}}

\begin{table}
	\makebox[\textwidth][c]{
	\begin{tabular}{r | c c c |}
local \textbackslash{} master & (\track, \same) & (\track, \modif) & (\track, \absent) \\
\hline
(\sub, \same) 
& (\sub, \donothing, \track, \donothing) & (\sub, \cpfile, \track, \metaupd) & (\subrm, \rmfile, \trackrm, \donothing)
\\
(\sub, \modif)
& (\sub, \metaupd, \track, \cpfile) & (\sub, \cplatest, \track, \cplatest) & $^1$[\sub, \track] / [\subrm, \trackrm]
\\
(\sub, \absent)
& $^2$[\unsub, \track] / [\subrm, \trackrm] & $^2$[\unsub, \track] / [\subrm, \trackrm] & (\subrm, \donothing, \trackrm, \donothing)
\\
\hline
	\end{tabular}
}
	\caption{State machine for tracked files}
\end{table}

\begin{table}
	\makebox[\textwidth][c]{
	\begin{tabular}{r | c c |}
local \textbackslash{} master & (\trackrm, \present) & (\trackrm, \absent) \\
\hline
(\subrm, \present)
& (\sub, \cplatest, \track, \cplatest) & (\sub, \metaupd, \track, \cpfile) \\
(\subrm, \absent)
& $^3$[\sub, \track] / [\unsub, \track] & (\subrm, \donothing, \trackrm, \donothing) \\
\hline
	\end{tabular}
}
	\caption{State machine for untracked files}
\end{table}

\begin{table}
	\makebox[\textwidth][c]{
	\begin{tabular}{r | c c |}
local \textbackslash{} master & (\track, \present) & (\track, \absent) \\
\hline
(\sub, \present) 
& (\sub, \donothing, \track, \donothing) & (\subrm, \rmfile, \trackrm, \donothing)
\\
(\sub, \absent)
& $^2$[\unsub, \track] / [\subrm, \trackrm] & (\subrm, \donothing, \trackrm, \donothing)
\\
\hline
(\unsub, \present)
& (\sub, \mergefolders, \track, \mergefolders) & $^1$[\sub, \track] / [\subrm, \trackrm]
\\
(\unsub, \absent)
& (\unsub, \donothing, \track, \donothing) & (\subrm, \donothing, \trackrm, \donothing)
\\
\hline
	\end{tabular}
}
	\caption{State machine for tracked folders}
\end{table}

\begin{table}
	\makebox[\textwidth][c]{
	\begin{tabular}{r | c c |}
local \textbackslash{} master & (\trackrm, \present) & (\trackrm, \absent) \\
\hline
(\subrm, \present)
& (\sub, \mergefolders, \track, \mergefolders) & (\sub, \donothing, \track, \cpfile) \\
(\subrm, \absent)
& $^3$[\sub, \track] / [\unsub, \track] & (\subrm, \donothing, \trackrm, \donothing) \\
\hline
	\end{tabular}
}
	\caption{State machine for untracked folders}
\end{table}

\begin{table}
	\makebox[\textwidth][c]{
	\begin{tabular}{r | l}
state & action \\
\hline
(\sub, \same) & \rmfile \\
(\sub, \modif) & Conflict \\
(\sub, \absent) & \donothing \\
\hline
(\subrm, \present) & Conflict \\
(\subrm, \absent) & \donothing \\
\hline
	\end{tabular}
}
	\caption{\rmfile{} mechanism for files. In case of conflicts, keep file and print warning}
\end{table}

\begin{table}
	\makebox[\textwidth][c]{
	\begin{tabular}{r | l}
state & action \\
\hline
(\sub, \present) & \action{recurse then rmdir} \\
(\sub, \absent) & \donothing \\
\hline
(\unsub, \present) & Conflict \\
(\unsub, \absent) & \donothing \\
\hline
(\subrm, \present) & Conflict \\
(\subrm, \absent) & \donothing \\
\hline
	\end{tabular}
}
	\caption{\rmfile{} mechanism for folders. In case of conflicts, keep folder and print warning}
\end{table}

\section{Conflicts}

\textit{TODO}

\section{Metadata}

\newcommand{\unix}[1]{\texttt{#1}}

Metadata is stored in every folder in a special \unix{.gimme} folder. Two possibilities from this.

Two subfolders \unix{folders} and \unix{files}. Every tracked file / folder gets its own file in the corresponding subfolder. This is inefficient (lots of files created), but well-suited for a Bash implementation.

Two files \unix{folders} and \unix{files}. Every tracked file / folder gets an entry in the corresponding file. This is more efficient, but requires more text processing. Would be rather Perl-friendly.

Optimization : 

\section{Algorithm}

\begin{lstlisting}[frame=ltbr, tabsize=4]
gimme(localdir, masterdir) {
	check number of arguments
	set up constants
	merge_folders(localdir, masterdir)
}

normalize_these(list, localmeta, mastermeta) {
	for x in list
	do
		if (x notin localmeta)
			echo RM > localmeta/x
			
		if (x notin mastermeta)
			echo RM > mastermeta/x
	od
}

normalize(localdir, masterdir) {
	localmetafolders = localdir/.gimme/folders
	mastermetafolders = masterdir/.gimme/folders
	localmetafiles = localdir/.gimme/files
	mastermetafiles = masterdir/.gimme/files
	
	ensure .gimme folder exists and correct shape

	normalize_these(folders(localdir), localmetafolders, mastermetafolders)
	normalize_these(folders(masterdir), localmetafolders, mastermetafolders)
	normalize_these(tracked_folders(localdir), localmetafolders, mastermetafolders)
	normalize_these(tracked_folders(masterdir), localmetafolders, mastermetafolders)
	
	normalize_these(files(localdir), localmetafiles, mastermetafiles)
	normalize_these(files(masterdir), localmetafiles, mastermetafiles)
	normalize_these(tracked_files(localdir), localmetafiles, mastermetafiles)
	normalize_these(tracked_files(masterdir), localmetafiles, mastermetafiles)
}

merge_folders(localdir, masterdir) {
	normalize(localdir, masterdir)
	
	for subfolder in tracked_folders(localdir)
	do
		merge_subfolder(subfolder, localdir, masterdir)
	od
	
	for file in tracked_files(localdir)
	do
		merge_file(file, localdir, masterdir)
	od
}

subfolder_state(subfolder, folder) {
	read_meta(subfolder, folder) + (subdir in folder ?)
}

compute_state(subfolder, localdir, masterdir) {
	localstate = subfolder_state(subfolder, localdir)
	masterstate = subfolder_state(subfolder, localdir)
	return localstate + masterstate
}
	
merge_subfolder(subfolder, localdir, masterdir) {
	s = compute_state(subfolder, localdir, masterdir)
	switch (s)
	{
		case ...
			ask to resolve conflict ?
			recursively call merge or rm folder
			write new state for this subfolder in both local and master
	}
}

file_state(file, folder) {
	read_meta(file, folder) + (file in folder ? updated ?)
}

compute_state(file, localdir, masterdir) {
	localstate = file_state(file, localdir)
	masterstate = file_state(file, localdir)
	return localstate + masterstate
}

merge_file(file, localdir, masterdir) {
	s = compute_state(file, localdir, masterdir)
	switch (s)
	{
		case ...
			ask to resolve conflict ?
			some conflicts just raise warning
			copy / update / rm file
			write new state both in local and master
	}
}


rm_folder(folder) {

	for subfolder in folders(folder)
	do
		rm_subfolder(subfolder, folder)
	od
	
	for file in files(folder)
	do
		rm_file(file, folder)
	od
	
	rm -r .gimme folder
}

rm_subfolder(subfolder, folder) {
	m = meta(subfolder, folder)
	if (m = UNSUB or m = RM)
		raise warning
		do nothing
	else
		rm_folder(subfolder)
		rmdir subfolder  // might fail due to non-empty, this is ok
}

rm_file(file, folder) {
	v = compute SAM or UPD
	m = meta(file, folder)

	if (v = UPD or m = RM)
		raise warning
		do nothing
	else
		rm file
}
\end{lstlisting}

\subsection{In Perl}

\textit{TODO}

\end{document}